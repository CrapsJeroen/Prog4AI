\section{Task 1: Sudoku}\label{sec:sudoku}
\subsection{Introduction}

An \textit{n-sudoku} puzzle contains $n^{4}$ squares, in an $n^{2}$ by $n^{2}$ grid, and has $n^{2}$ blocks. These blocks are of size $N$ by $N$. Sudoku puzzles are proven to be \textbf{NP-Complete}. The collection of these rows, columns and zones are all units in the puzzle. So we come to the following statement:

\begin{center}
\textit{A puzzle is solved if every unit contains every element \\ from the interval $1$ to $n^2$ exactly once.}
\end{center}

This means that every unit is filled with a permutation of $[1,2,\cdots,n^{2}$. The classical viewpoint for Sudoku states that all numbers in a row must be different, that all numbers in a column must be different, and that all numbers in a block must be different.\\

\begin{center}
\begin{tabu}{|[2pt] c | c | c |[2pt] c | c | c |[2pt] c | c | c |[2pt]} \tabucline[2pt]{-}
	$x_{1,1}$ & $x_{1,2}$ & $x_{1,3}$ & $x_{1,4}$ & $x_{1,5}$ & $x_{1,6}$ & $x_{1,7}$ & $x_{1,8}$ & $x_{1,9}$ \\ \hline
	$x_{2,1}$ & $x_{2,2}$ & $x_{2,3}$ & $x_{2,4}$ & $x_{2,5}$ & $x_{2,6}$ & $x_{2,7}$ & $x_{2,8}$ & $x_{2,9}$ \\ \hline
	$x_{3,1}$ & $x_{3,2}$ & $x_{3,3}$ & $x_{3,4}$ & $x_{3,5}$ & $x_{3,6}$ & $x_{3,7}$ & $x_{3,8}$ & $x_{3,9}$ \\ \tabucline[2pt]{-}
	$x_{4,1}$ & $x_{4,2}$ & $x_{4,3}$ & $x_{4,4}$ & $x_{4,5}$ & $x_{4,6}$ & $x_{4,7}$ & $x_{4,8}$ & $x_{4,9}$ \\ \hline
	$x_{5,1}$ & $x_{5,2}$ & $x_{5,3}$ & $x_{5,4}$ & $x_{5,5}$ & $x_{5,6}$ & $x_{5,7}$ & $x_{5,8}$ & $x_{5,9}$ \\ \hline
	$x_{6,1}$ & $x_{6,2}$ & $x_{6,3}$ & $x_{6,4}$ & $x_{6,5}$ & $x_{6,6}$ & $x_{6,7}$ & $x_{6,8}$ & $x_{6,9}$ \\ \tabucline[2pt]{-}
	$x_{7,1}$ & $x_{7,2}$ & $x_{7,3}$ & $x_{7,4}$ & $x_{7,5}$ & $x_{7,6}$ & $x_{7,7}$ & $x_{7,8}$ & $x_{7,9}$ \\ \hline
	$x_{8,1}$ & $x_{8,2}$ & $x_{8,3}$ & $x_{8,4}$ & $x_{8,5}$ & $x_{8,6}$ & $x_{8,7}$ & $x_{8,8}$ & $x_{8,9}$ \\ \hline
	$x_{9,1}$ & $x_{9,2}$ & $x_{9,3}$ & $x_{9,4}$ & $x_{9,5}$ & $x_{9,6}$ & $x_{9,7}$ & $x_{9,8}$ & $x_{9,9}$ \\ \tabucline[2pt]{-}
\end{tabu}
\captionof{figure}{Example of the classical viewpoint on a $3$-sudoku puzzle.}
\end{center}

We can now define the Sudoku problem formally. A value at position row $i$ and column $j$ in the grid is represented as $x_{i,j}$. We define the following constraints for an \textit{n-sudoku} puzzle:

\begin{center}
\begin{tabular}{l l l}
\textbf{Variables}: & $ x_{1,1},x_{1,2},\cdots,x_{i,i},x_{i,i+1},\cdots,x_{n^{2},n^{2}} \in \left\{1,2,\cdots,{n^2}\right\}$ & \\
\textbf{Constraints}: & $\forall i, j, k \in \left\{1,2,\cdots,n^{2}\right\}: x_{i,k} \neq x_{j,k}$ & \textbf{Rows}\\
& $\forall i, j, k \in \left\{1,2,\cdots,n^{2}\right\}: x_{i,j} \neq x_{i,k}$ & \textbf{Columns}\\
& $\forall i, j \in \left\{0,1,\cdots,n-1\right\}, \forall a, b, c, d \in \left\{0,1,\cdots,n\right\} : x_{i*n+a,j*n+b} \neq x_{i*n+c,j*n+d}$ & \textbf{Blocks}\\
\end{tabular}
\end{center}

\subsection{Alternative viewpoint}

In this section we will discuss an alternative viewpoint for an classical \textit{n-sudoku} puzzle.
Now we will define the puzzle as a problem of $n^{2}$ rows, so we group all the elements in a row into a single variable. 
These rows will be permutations of the list $[1,2,\cdots,n^{2}]$. 
The reasoning behind this viewpoint is to reduce the amount of constraints needed to express the problem. 
The amount of variables need has been reduced from $n^{4}$ to $n^{2}$. \\

Again we can define this more formally. We define $n^{2}$ variables $r_i$ instead of the $n^{4}$ variables in the classical sudoku viewpoint. \textbf{TODO!!!}
So to visualize this: 

\begin{center}
\begin{tabu}{ |[2pt] l c r |[2pt]}\tabucline[2pt]{-}
& $r_{1}$ & \\ \hline
& $r_{2}$ & \\ \hline
& $r_{3}$ & \\ \tabucline[2pt]{-}
& $r_{4}$ & \\ \hline
& $r_{5}$ & \\ \hline
& $r_{6}$ & \\ \tabucline[2pt]{-}
& $r_{7}$ & \\ \hline
& $r_{8}$ & \\ \hline
& $r_{9}$ & \\ \tabucline[2pt]{-}
\end{tabu}
\captionof{figure}{Example of the alternative viewpoint on an 3-sudoku puzzle.}
\end{center}


\begin{center}
\begin{tabular}{l l l}
\textbf{Variables}: & $ r_{1},r_{2},\cdots,r_{n^{2}} \in \{[1,2,\cdots,n^{2}],[2,1,\cdots,n^{2}],\cdots,[n^{2},n^{2}-1,\cdots,1]\}$ & \\
\textbf{Constraints}: & $\forall i, j, k \in \left\{1,2,\cdots,n^{2}\right\}: r_{i,k} \neq r_{j,k}$ & \textbf{Columns}\\
& $\forall i, j \in \left\{0,1,\cdots,n-1\right\}, \forall a, b, c, d \in \left\{0,1,\cdots,n\right\} : r_{i*n+a,j*n+b} \neq r_{i*n+c,j*n+d}$ & \textbf{Blocks}\\

\end{tabular}
\end{center}

\subsection{Criteria}
The criteria according to us for a good viewpoint are the following: an easy understanding of the representation, the computational complexity to get to a solution, the amount of backtracks or logical inferences that are necessary to get a solution from the implementation of the proposed viewingpoint.
In our opinion the alternative viewpoint that we suggested is a good viewpoint, because the amount of backtracking that is necessary is greatly reduced by reducing the amount of variables in the problem and restricting their domain. 

\subsection{Channeling Constraints}
The channeling constraints between the classical viewpoint and the alternative viewpoint can only be defined in one direction.
It is only possible to define the values of the individual cells in the classical viewpoint when coming from the rows in the alternative viewpoint.
If a row is assigned a certain permutation as its value, then the elements of the corresponding row in the original viewpoint can be assigned the values from this permutation.
In the other direction this is not possible unless all of the variables in a row have been assigned a value and that they are all different from each other.
So that these values from a permutation of the original list.
The alternative viewpoint has as main advantage that the row-constraints are implied when declaring its domain.
The influence of this is mainly noticed in the search process, instead of instantiating every variable separatly the entire row will be instantiated at once.
Due to these channeling constraints being so trivial we think that there is no point in implementing them, since the influence would be minimal.  

\subsection{Implementation}
\subsubsection{ECLiPSe}
\paragraph*{Original viewpoint}
In the original viewpoint we view the sudoku as a matrix with $n^{4}$ variables.
The constraints on this are the following.
We extract the required information from this matrix by defining rows, columns and blocks.
For each row, column and block we say that every value in these lists must be different.
So all of the constraints are exactly the same, but the way of extracting this information from the matrix is different.\\

A row with index $I$ is defined in a matrix by $Row is Sudoku[I,1..N^{2}]$. 
For extracting a column with index $I$ we use $Col is Sudoku[1..N^{2},I]$.
While extracting the information for the rows and columns is straight forward, it becomes clear when extracting the variables required to fulfill the block constraints.
We define the block with index $(I,J)$ as follows. 
The indices $I, J$ are elements from the interval $[0,1,\cdots,(n-1)]$.
The block at the upperleft corner of the puzzle will be named the $(0,0)$-block and the block at the bottomright corner will be the $(n-1,n-1)$-block.
For each pair $(I,J)$ we now need to define the list containing its $n^{2}$ elements.
We need two new variables $(K,L)$ to define the rows and columns needed to form these blocks.
For example, the $X$ coordinate in our matrix will be equal to $(I*n)+K$. 
So for the $(0,0)$ the required rows are $1,2,\cdots,n-1$.
A similar approach can be made for the columns where $Y = (J*n)+L$.
Combining all of the possible $X$ and $Y$ values based on the given $(I,J)$ pair leads to the all of the indices required to create the specified $(I,J)$-block.

\begin{lstlisting}
( multifor([I,J],0,(N-1)), param(Sudoku, N) do
	( multifor([K,L],1,N), param(Sudoku, I, J, N), foreach(B,Block) do
        	R is (I*N)+K,
        	C is (J*N)+L,
		B is Sudoku[R,C]
    ),
	alldifferent(Block)
).
\end{lstlisting}


\paragraph*{Alternative viewpoint}

\subsubsection{CHR}
\subsubsection{Decision}

\subsection{Experiments}
\subsubsection{Results}
\subsubsection{Heuristics}
\subsubsection{Difficulties}

\subsection{Conclusions}


